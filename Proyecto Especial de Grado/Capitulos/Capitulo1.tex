% Capítulo 1

\chapter{El problema} % Título Principal del Capítulo

\label{Capitulo1} % Para hacer referenciar este capítudo use \ref{Capitulo1}

%----------------------------------------------------------------------------------------
%	SECTION 1
%----------------------------------------------------------------------------------------

\section{PLANTEAMIENTO DEL PROBLEMA.}

 En pleno siglo XXI la población automotriz ha aumentado considerablemente, según un censo mundial proporcionado en el 2015 por la agencia \textit{“Organisation Internationale des Constructeurs d’Automobiles”} (OICA), la patronal mundial del motor refleja una cifra de mas de 1200 millones de automóviles en todo el mundo. A causa de ello el estudio del tr\'ansito terrestre es muy importante  cuando se requiere realizar la remodelación o construcción de una vialidad, elevado o cualquier infraestructura ya que dicho estudio proporciona información del volumen de transito que circulan por una vía publica y en consecuencia  podrá saberse a ciencia cierta como sera el impacto generado en el tr\'ansito vehicular a causa de lo mencionado con anterioridad. El aforo vehicular esta vinculado directamente con el  EIV(ESTUDIO DEL IMPACTO VIAL) dado que otorga un histograma de la cantidad de vehículos clasificados según el tipo que circulan cierta cantidad de horas al día durante un tiempo determinado, donde el mismo depende de la cantidad de días que se realice el conteo vehicular.

Ahora bien el conteo de vehículos anteriormente era realizado de manera manual lo que traía como consecuencia que se generaran errores debido las limitaciones del grupo de personas encargados del conteo, en vista de que se han realizado investigaciones en cuanto a la creación de algoritmos de conteo vehicular y a los avances de la visión por computador, los mismos han permitido que se creen programas que determinen el TPDA(Tránsito promedio diario anual) de forma digital facilitando de esta manera el conteo de vehículos  y por ende el error generado debido al conteo manual.

{\noindent BLANCO y HERNANDEZ (2016)[2] afirman:}

{\setlength{\parindent}{1.5em}En cuanto a las herramientas que suelen ser utilizadas para solventar los inconvenientes generados del conteo manual, la vídeo detección ofrece la mayor cantidad de ventajas, obteniendo mejor resultado en el reconocimiento vehicular; siendo las cámaras una de las herramientas más simples de instalar, ubicadas en la categoría de sensores no intrusivos, es decir, no afecta al tránsito durante su instalación o medición, permitiendo una detección óptima y una solución económica. En otras palabras esta es una herramienta versátil que le permite al usuario simplificar el trabajo y disminuir el tiempo de procesamiento impulsada por el avance tecnológico en el área de la visión por computador, además de contar vehículos, proporcionar datos como velocidad, tipo de vehículo, densidad, reconocimiento de placa, entre otros.}

En relación con el conteo vehicular de manera automatizada, en la universidad de Carabobo se realizo un sistema de aforo vehicular automatizado titulado "DESARROLLO DE UN SISTEMA DE AFORO VEHICULAR MEDIANTE PROCESAMIENTO DIGITAL DE VÍDEO" donde se plantearon algunas recomendaciones las cuales son plasmadas a continuación:

\begin{enumerate}
	\item Modificar el sistema desarrollado para ser utilizado en vídeos en vivo.
	\item Buscar otra forma de delimitar los intervalos de aforo y así distribuir en el tiempo correctamente los vehículos aforados, de manera que no se requiera de un vídeo con valores de cuadros por segundo constantes.
	\item Añadir al sistema un módulo de reconocimiento que sea capaz de monitorear, detectar y clasificar los resultados del módulo de detección con la finalidad de determinar si un vehículo corresponde a la clasificación de liviano, bus, pesado entre otros.
	\item Modificar el sistema desarrollado para disminuir el tiempo de aforo, implementando procesamiento en paralelo. De esta forma se podría analizar vídeos de larga duración en fracciones de su tiempo.
	
\end{enumerate}

Con lo expuesto anteriormente damos a entender que partiremos de las recomendaciones planteadas en el trabajo de grado titulado "DESARROLLO DE UN SISTEMA DE AFORO VEHICULAR MEDIANTE PROCESAMIENTO DIGITAL DE VÍDEO" permitiendo realizar una actualización del mismo y por ende obtener un software mas potente. 
 

%-----------------------------------
%	SUBSECTION 1
%-----------------------------------
\section{JUSTIFICACIÓN DE LA INVESTIGACIÓN.}

El aforo vehicular manual usado en el EIV contiene algunas desventajas dentro de las cuales puede mencionarse el dinero usado en la contratación del personal encargado de efectuar el conteo vehicular, el error generado debido a la distracción por parte del personal, el tiempo invertido en dicho conteo; también debe ser tomado  en cuenta el hecho de que en el trabajo de grado titulado: DESARROLLO DE UN SISTEMA DE AFORO VEHICULAR MEDIANTE PROCESAMIENTO DIGITAL DE VÍDEO realizado en la FACULTAD DE INGENIERÍA DE LA UNIVERSIDAD DE CARABOBO dejo plasmadas unas recomendaciones confirmando que puede realizarse una actualización del mismo creando de esta manera un software  m\'as poderoso. Es por ello que debe realizarse una actualización del SOFTWARE SISTEMA DE AFORO VEHICULAR MEDIANTE PROCESAMIENTO DIGITAL DE VÍDEO permitiendo así un EIV mas efectivo y fiable de manera automatizada, que tome en cuenta la clasificación del vehículo según su forma para así  determinar el grado de ocupación y las condiciones en que opera cada segmento de la vía p\'ublica, otro elemento importante a tomar en consideración es el tiempo usado en el conteo en donde el mismo debe ser lo mas corto posible por lo cual se implementara un procesamiento en paralelo con lo que se podría procesar vídeos de larga duración en fracciones de su tiempo. 
%-----------------------------------
%SECTION 2
%-----------------------------------

\section{OBJETIVOS.}
\subsection{Objetivo General}
ACTUALIZAR EL SOFTWARE SISTEMA DE AFORO VEHICULAR MEDIANTE PROCESAMIENTO DIGITAL DE VÍDEO.
\subsection{Objetivos Específicos}
\begin{enumerate}
	\item Evaluar el software " SISTEMA DE AFORO VEHICULAR MEDIANTE PROCESAMIENTO DIGITAL DE VÍDEO ".
	\item Realizar las mejoras del Software para el conteo de vehículos automatizado.
	\item Evaluar y validar el software diseñado.
	
\end{enumerate}

\section{ALCANCES}
La ACTUALIZACIÓN DEL SOFTWARE SISTEMA DE AFORO VEHICULAR MEDIANTE PROCESAMIENTO DIGITAL DE VÍDEO permitirá realizar conteo de vehículos procesando vídeos en vivo, así como clasificar los automóviles según el tipo ya sea pesado, bus, liviano entre otros. El programa diseñado sera multiplataforma y autoejecutable, la instalación del mismo sera intuitiva para cualquier usuario en general.

La versión mejorada del SOFTWARE SISTEMA DE AFORO VEHICULAR MEDIANTE PROCESAMIENTO DIGITAL DE VÍDEO podrá distribuir en el tiempo correctamente los vehículos aforados, de manera que no se requiera de un vídeo con valores de cuadros por segundo constantes. También tendrá la capacidad de analizar vídeos de larga duración en fracciones de su tiempo y así poder disminuir el tiempo de aforo debido al procesamiento en paralelo. Este software  sera usado por dos   entes gubernamentales, la Alcaldía de San Diego y la Alcaldía de Maracay para así constatar la calidad del mismo.